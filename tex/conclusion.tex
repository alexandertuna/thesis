\chapter[Conclusions][Conclusions]{Conclusions}
\label{chap:conclusion}

This thesis described evidence of Higgs boson decays to tau leptons with the ATLAS detector at the LHC, with special emphasis given to the VBF $\Htautaulh$ subset of the analysis. The theoretical context, LHC, and ATLAS experiment were briefly reviewed. The signature of tau leptons at ATLAS was described in detail.

The data in the $\Htautau$ analysis correspond to $25\ \ifb$ of proton collisions with $\sqrt{s} = $ 7 or 8 TeV. Strong evidence for $\Htautau$ is observed (expected) with a $4.5\sigma$ ($3.4\sigma$) deviation from the background-only hypothesis. The measured signal strength, normalized to the Standard Model expectation, is $1.4^{+0.4}_{-0.4}$, which is consistent with the Standard Model prediction. A limiting factor of the measurement is the size of the available dataset.

Future LHC data-taking campaigns will offer substantially more data and at a higher collision energy, though the harsh conditions present challenges for triggering on $\tauh$ and rejecting pileup jets mimicking the VBF signature. The VBF $\Htautaulh$ analysis projects to measure a signal strength uncertainty of 8\% with the addition of a high performance, high coverage forward tracker.


% \cref{chap:standardmodel} gives a description of the Standard Model of particle physics to provide theoretical context for searches for the Higgs boson. \cref{chap:lhcatlas} describes the LHC and the ATLAS detector, which are the experimental apparatuses used here. \cref{chap:taus} describes tau leptons and their experimental signatures at ATLAS.

% \cref{chap:strategy} describes the strategy for searching for $\Htautau$ at ATLAS. \cref{chap:backgrounds} reviews how physics processes relevant to the search are predicted, with special emphasis given to mis-identified $\tauh$. \cref{chap:results} gives the results of the searches. \cref{chap:prospects} concludes this thesis with a discussion of future prospects for $\Htautau$ analysis at ATLAS, both in the near- and long-term.


