\chapter[Introduction][Introduction]{Introduction}
\label{chap:introduction}

This thesis documents the evidence of Higgs boson decays to tau leptons with the ATLAS experiment at the LHC. Special emphasis is given to the VBF $\Htautaulh$ subset of the analysis. The data correspond to $25\ \ifb$ of proton collisions with $\sqrt{s} = $ 7 or 8 TeV.

\cref{chap:standardmodel} gives a brief overview of the Standard Model of particle physics to provide theoretical context for searches for the Higgs boson. \cref{chap:lhcatlas} describes the LHC and the ATLAS detector, which are the experimental apparatuses used here, and the process by which detector outputs are interpretated and classified as particles. \cref{chap:taus} describes tau leptons and their experimental signatures at ATLAS. Details of classifying hadronic tau lepton decays are presented, especially discrimination against jets, electrons, and muons.

\cref{chap:strategy} discusses the search strategy for $\Htautau$ at ATLAS and motivates the use of machine learning. \cref{chap:backgrounds} reviews how physics processes relevant to the search are predicted, including a thorough description of the prediction of jets mis-identified as hadronic tau lepton decays. \cref{chap:results} gives the results of the searches and presents evidence for decays of the Higgs boson to tau leptons at ATLAS.

\cref{chap:prospects} discusses future prospects for $\Htautau$ analysis at ATLAS, both in the near- and long-term. \cref{chap:conclusion} concludes this thesis with a synopsis.

Much of the work in this thesis is included in publications by the ATLAS experiment, including a description of hadronically decaying tau leptons at ATLAS~\cite{PERF-2013-06} and evidence for decays of the Higgs boson to tau leptons at ATLAS~\cite{HIGG-2013-32}.

