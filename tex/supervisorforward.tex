The development of spontaneously broken gauge theories in the 1960’s  both unified our understanding of electromagnetic and weak interactions and promised an understanding of how and why particles acquire mass. While the predicted $W$ and $Z$ bosons were observed in 1982, the Higgs boson, which is expected to be responsible for the generation of masses of both the vector bosons and quarks and leptons, has remained elusive.  The theory of Supersymmetry, which hypothesizes the existence of partners of all the known particles, emerged in the 1970’s and early 1980’s both to explain the “hierarchy problem” (why the Higgs mass could remain low) and as a necessary ingredient of String Theory which unifies gravity and the other forces.

	The Large Hadron Collider (LHC) was designed to enable discovery of the Higgs boson, if it exists, and also to provide searches of vastly increased sensitivity for Supersymmetric particles (as well as other exotic new states). Decay of the Higgs into two photons has long been considered one of the most promising final states if it’s mass is less than $\sim$130 GeV. Similarly, one and two photon final states with missing transverse energy provide good sensitivity to Supersymmetry in the case of gauge mediated symmetry breaking.  The two largest experiments at the LHC, ATLAS and CMS, were designed with these searches high on the list of priorities.

	 The ATLAS experiment consists of an inner tracking detector  immersed in a 2 Tesla solenoidal magnetic field, and surrounded by a highly segmented liquid argon calorimeter, scintillating tile calorimeter, and precision muon chambers.  The inner tracking detector is composed of several layers of pixel sensors and silicon strip detectors, followed by a Transition Radiation Tracker (TRT) composed of 300,000 4 mm diameter straw proportional tubes. The TRT is a critical component of the overall inner detector, improving the track reconstruction,  providing excellent momentum resolution, and aiding in electron identification via detection of transition radiation. A section of this thesis describes the electronics and data acquisition of the TRT to which Michael Hance contributed enormously over a period of nearly five years.

	The measurement of high energy photons in a colliding beams environment is quite challenging: background from jets which fragment to a single leading \pizero\ or $\eta$ meson provide a formidable background.   Even after carefully chosen cuts on shower shape variables reduce the probability that a jet fakes a photon by several thousand, how does one determine the purity of the sample selected?  Finally, the absence of a narrow, massive resonance decaying to two photons makes it difficult to measure efficiencies. 

	Michael was one of a small group of physicists who tackled photon identification in ATLAS from the start of first collisions. He focused especially on the fact that photons from a Higgs boson or sparticle decay, as well as from  most “hard scattering processes”, are isolated, i.e. there is almost no associated particle activity in an angular cone around the direction of the photon. By demanding that the photon be “isolated”, one can further reduce the background from jets, and by measuring the distribution of energy in the “isolation cone” after all other selection criteria one can determine the purity of the final sample. While this technique had been used already in prior experiments, for example in CDF, these earlier efforts relied heavily on Monte Carlo simulation to obtain the “expected isolation distributions”.  Through a combination of careful and insightful work, Michael was able to determine reliably, from the data itself, the isolation distributions for both background and signal. Along the way, he determined that more energy was leaking outside the “EM core” than had been thought and also developed a rather powerful way of estimating, and subtracting off, energy in the isolation cone from the “underlying event”.  He then used these isolation templates to perform fits to the observed isolation distributions for single, inclusive photons and contributed greatly to the first measurement of the inclusive photon cross section at ATLAS. This thesis describes, first and foremost, the above work.

	The techniques developed and presented in this thesis were rapidly adopted by groups measuring the diphoton cross section, by those searching for Higgs to gamma gamma, and by others searching for new physics with photons. The techniques were also adopted for improving the purity of electron selection, and for measuring residual background in final states with electrons.  The conclusion of the thesis gives a first look at purity measurements for diphotons in the context of the Higgs search. 

As of the writing of this foreword, the ATLAS experiment has presented strong evidence of a resonance in the diphoton final state that may prove to be the Higgs boson. The work described in this thesis, both for photon identification and for the operation of the TRT, played a critical role in this observation.


\vspace*{2mm}

Professor H.H. Williams, University of Pennsylvania. June, 2012
